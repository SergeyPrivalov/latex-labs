\Large

{ 
  % \it % должно быть, но не будет

  \indent Последовательность называется \textbf{МОНОТОННОЙ}, если она -
  или возрастающая, или убывающая, причем $(x_n)_{n=1}^{\infty}$ -
  \textbf{возрастающая} (в строгом смысле), если, начиная с некоторого
  номера, \textbf{каждый} последующий ее член больше (строго)
  предыдущего члена, т.е.
  \begin{equation*}
    ((x_n)_{n=1}^{\infty} \uparrow) \leftrightarrow
    (\exists n_0 : \forall n > n_0 : x_n < x_{n+1});
  \end{equation*}
  аналогично
  \begin{equation*}
    ((x_n)_{n=1}^{\infty} \downarrow) \leftrightarrow
    (\exists n_0 : \forall n > n_0 : x_n > x_{n+1}).
  \end{equation*}

  Иногда рассматривают возрастание и убывание
  последовательности в нестрогом смысле.

  Последовательность называется \textbf{СТАЦИОНАРНОЙ}, если,
  начиная с некоторого номера, \textbf{все} члены последовательности
  совпадают, т.е.
  \begin{equation*}
    ((x_n)_{n=1}^{\infty} \text{- стационарная}) \leftrightarrow
    (\exists n_0; \forall n > n_0 : x_n = x_{n + 1}).
  \end{equation*}

  Для обоснования монотонности последовательности можно:
  \begin{itemize}
    \item либо установить сохранность знака разности $x_{n+1} - x_n$ для всех $n$, начиная с некоторого;
    \item  либо убедиться в выполнимости неравенства $\frac{x_{n+1}}{x_n} < 1$ (или $>1$)
    для всех $n$, начиная с некоторого.
  \end{itemize}
}
Например, последовательность $(\frac{2n+1}{n+2})_{n=1}^{\infty}$ - возрастающая
(в строгом смысле), поскольку можно воспользоваться соотношением
либо 
$x_{n+1} - x_n = 
\frac{2(n+1)+1}{(n+1)+2} - \frac{2n+1}{n+2} = 
\frac{2n+3}{n+3} - \frac{2n+1}{n+2} = 
\frac{1}{(n+2)(n+1)} > 0$ для всех $n \in N$,
либо 
$\frac{x_{n+1}}{x_n} = 
\frac{2n+3}{n+3} : \frac{2n+1}{n+2} =
\frac{(2n+3)(n+2)}{(n+3)(2n+3)} =
\frac{2n^2+5n+3}{2n^2+5n+2} > 1$ для всех
$n \in N$.
